\documentclass[12pt, a4paper]{article}

\usepackage{amssymb}
\usepackage{multicol}
\usepackage{enumerate}
\usepackage[top=5em, bottom=5em, left=5em, right=5em]{geometry}
\usepackage{listings}
\usepackage{tikz}
\usetikzlibrary{positioning}

\setlength\parskip{1em}
\setlength\parindent{0em}

\title{Assignment}

\author{Hendrik Werner s4549775}

\begin{document}
\maketitle

This was done in collaboration with Constantin Blach (s4329872).

\section{} %1
We came up with the following algorithm:

\begin{enumerate}
	\item Split the highway into sections $S = \{s_1, \dots, s_m\}$ where the last stop in $s_i$ is at least $k$ kilometers apart from the first stop in $s_{i + 1}$.
	\item For each of those sections calculate the optimal expected profit of that section.
	\label{sc1:solution_per_section}
	\item Sum solutions for all sections.
\end{enumerate}

You can find the optimum of each solution at step \ref{sc1:solution_per_section} by recursively trying how much you earn with and without this stop at any stop you visit, then returning the maximum you can get. You start this at the fist node of each section and get

\paragraph{Correctness}
All stops in one section $s_i$ are at least $k$ kilometers apart from any stop $s_j, s_k, j < i, i < k$. Thus all sections are independent of each other and can be computed separately as they cannot interfere with each other

\paragraph{Time Complexity}
The worst case is that the highway cannot be split into separate sections because all consecutive stops are less than $k$ kilometers apart. In this case we can visualize the nodes we visit at a tree. The left child of each node is the next stop in order, the child to its right is the next stop at least $k$ kilometers away.

We always take the full path to the left which has a depth of $O(\log n)$ and we have a maximum of $O(n)$ of these "left-paths". The time complexity is therefore $O(n \log n)$.

\section{} %2
\begin{enumerate}[a]
	\item %a
	We came up with the following algorithm which I implemented in Groovy:

	\lstinputlisting{code/findCPathsRecursive.groovy}

	It can be used like this

	\begin{lstlisting}
int[][] matrix = [[1, 2, 3], [4, 6, 5], [3, 2, 1]]
println findCPaths(matrix, 12, 2, 2)
	\end{lstlisting}

	which produces the correct result $2$.

	\paragraph{Correctness}

	By taking both a step to the right and a step down at each position, we exhaustively try all possible paths to $M[m][n]$ so we are guaranteed to find the correct solution by summing all paths we find together.

	\paragraph{Time Complexity}

	\begin{itemize}
		\item At each position we visit we make a maximum of $C$ steps, beginning to the right, and a maximum of $C$ step beginning down: $O(2C) = O(C)$.
		\item We visit $m * n * 2C$ positions: $O(mn2C) = O(mnC)$.
	\end{itemize}

	The total time complexity is $O(mnC^2)$. It could drastically improved by dynamic programming as we see below.

	\item %b
	(We did not quite understand this part: "The algorithm should rely on the bottom-up computation of P [i, j, k] in a". What a? Where is a defined? We just found a bottom up algorithm with the correct time complexity.)

	The number of C-paths from $M[0][0]$ to $M[m][n]$ can be computed bottom up by traversing $M$ from top left to bottom right until you reach $M[m][n]$. At each $M[i][j]$ set $V[i][j]$ to the union of $V[i - 1][j]$ with $V[i][j + 1]$ and add the current value to all elements of $V[i][j]$. For every $v \in V[i][j], v = C$ increment $P[i][j][C]$.

	\paragraph{Correctness}

	We can only go right or down so at position $i, j$ we could have only come from $i - 1, j$ or $i, j - 1$. $V[i][j]$ contains all values you could possibly have when you reach point $M[i][j]$. If the values match $C$ then they represent C-paths.

	\paragraph{Time Complexity}

	We visit $mn$ nodes and do a maximum of $C$ word at each (taking union of two lists and adding to each), which means that the total time complexity is $O(mnC)$.

\end{enumerate}

\section{} %3
\begin{enumerate}[a]
	\item %a
	Here is a python implementation of our algorithm:

	\lstinputlisting{code/palindrome_top_down.py}

	Which prints the expected result $5$.

	\paragraph{Correctness}

	At every step our algorithm find the optimum of all possible steps it can take so it always finds the optimal solution.

	\paragraph{Time Complexity}

	The worst case is that the longest palindrome has length $1$. In this case we look at each letter twice: $O(2n) = O(n)$.

	\item %b
\end{enumerate}

\end{document}
